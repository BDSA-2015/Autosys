\section{Usability Testing}
To get a better understanding on the user of the Autosys system and their usage of the "Study Configuration UI", one must perform usability tests to get their feedback. It is worth to mention the design goal: Ease of use where a study configuration should be relatively easy to setup since it makes up the foundation that all study work processes rely on. (See System Design Document, Section 1.2 Design Goals).\\\\
At the initial  design of the user interface various paper mocks up will be created and used to conduct scenario tests combined with think-aloud tests. The user will perform pseudo actions on the mocks while explaining the testing team what they are doing and why they are doing it. This enables the testing team to notice whether a given mock is living up to given usability tasks and if there are any usability defects that must be addressed. This will be conducted as an iterative process until some final mockups have been created.  \\\\A final mock will be chosen and will then be implemented. In some given point of time, a prototype will be created and used for conducting prototype tests.  For this project, a horizontal user interface prototype will be used. This enables the test team to have a notion on how users interact with an actual interface and can thus address issues that may occur.  The test team may want to test if the system is fit for use, ease of learning and task efficiency. Fit for use because it is important to check whether the user can perform all of his tasks. Ease of learning is to check how well the user is able to interact with the interface correctly and task efficiency is to check how well the user is performing his tasks. All of these usability measures are then taken into consideration and used to optimize the user interface. \\\\When the system has most of its functionalities  implemented, a product test will be conducted to test if the implemented functionalities are working as requested.



