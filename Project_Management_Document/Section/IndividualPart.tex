\section{Individual Reflection}

\subsection{Individual Reflection of the Work Thor}
The thing that struck me the most, was the changed structure of the project. We have been used to work on projects with predefined tasks and requirements. As opposed to this project where we rely on our own observations and efforts. Initially, I did not understand that we were supposed to treat our teaching assistants and lecturer as people from within the application domain (e.g. consultants). I was rather skeptical about this approach but now understand how important it to be able to analyze customer needs in a given domain and derive a solution from these requirements. It may feel like a ”conjured” environment, since we do not keep contact with actual customers but the similarity is close and has helped challenge my work principles. 
\\
In regards to the group work, we started out with a flexible meeting schedule, which worked most of the time but I personally prefer a more planned approach. Sometimes we did not manage to review the work of all team members, which ultimately resulted in assignments with missing or inconsistent content. Consequently, we ended up using Trello to achieve a better overview along with a time schedule that all team members agreed with. This helped improve the cooperation and coordination of the project dramatically. 
\\
As in the prior projects, we used version control (Git) allowing us to work simultaneously on the mandatory assignments and code. This worked very well but required a strategy on how to use Git in order to avoid merge conflicts. Thus, we thoroughly went through on how to use branches and specific naming conventions when working on separate parts of an assignment or code project. I think we should have done this from the start, instead of having many merge conflicts requiring refactoring. This along with proper coding conventions minimize the overall time spend refactoring and makes what we produce more consistent and correct from the start. Consequently, I personally wrote a document about naming conventions and branches in Git, which has been read and understood by all team members. I had a bad experience with the lack of these agreements in the First Year Project were people would have different opinions about documentation, placement of brackets and version control. This ultimately resulted in many extra hours of work close to the deadline, which could have been avoided. 
\\
The work load was equally distributed among team members and I did not find any personal issues working together. This is based on previous experience working with the same people. Thus, we already know our different personalities and have thoroughly established the different skills that we have to complement each other. The only new conflict we had in this specific project was based on disagreements about time and priorities. In other words, some members were fine working on the project all days of the week while some members preferred a clearer separation between work related activities and social life. We had to solve this by making an actual document showing which times of the week each member is available or does not want to be disturbed. This will probably affect my choice of group for the next project, since I think a general agreement of work hours and time spent on the project is important to establish a common goal and work ethic. 

\subsection{Individual Reflection William}
The major issues I have encountered during the project was in regards to:
\begin{enumerate}
	\item Time
	\item Understanding the application domain
	\item Working with LaTex using Version Control
	\item Partly Unstructured Work and lacking Communication
	\item Hand-In of Assignments
\end{enumerate}
In regards to working on LaTex, I found that the communication and the rules for using git was too vague. This resulted in many merge conflicts and lost data, because the guidelines were not clear enough.
The work was done without enough structured planning in the beginning, which resulted in people not being sure about who was assigned to which specific assignment. Instead people had a more general idea about which parts of the assignments they were working on. This made it hard to know exactly who was responsible for what, and so it was hard to know, who to go to when having/spotting a problem. About the communication, the lack of this (also in the beginning) resulted in missing hand ins or hand ins with missing sections or wrong/duplicated UML diagrams.
Also the documentation of the development could have been better if versioning had been done in the beginning of the project. New realeses of the program and the different design documents should have been made frequently during the work to reflect design choices and evolution of the program better.\\
Another major struggle has been understanding the application domain. In the beginning the application domain seemed very confusing, but not enough effort were made to generate questions for the domain expert to explain the domain. This was a huge mistake, which had a seriouse impact on the project during the implementation, because the development of the program went in a wrong direction. This resulted in a wrong implementation, which had to be chaged quickly 2 days before the handin deadline. Because of this no functioning implementation of the program was ready for handin.

\subsection{Challenges in the Cooperation and Coordination}
Some troubles were encountered in accordance to when and how much to work. This was because there were conflicting opinions about how or even if it should be acceptable to declare oneself for unreachable because one would like to separate the study from the social life.
\pagebreak

\subsection{Individual Reflection Dennis}

\pagebreak
\subsection{Individual Reflection Jacob}