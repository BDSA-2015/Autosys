\section{Individual Reflection}

\subsection{Individual Reflection of the Work Thor}
The thing that struck me the most, was the changed structure of the project. We have been used to work on projects with predefined tasks and requirements. As opposed to this project where we rely on our own observations and efforts. Initially, I did not understand that we were supposed to treat our teaching assistants and lecturer as people from within the application domain (e.g. consultants). I was rather skeptical about this approach but now understand how important it to be able to analyze customer needs in a given domain and derive a solution from these requirements. It may feel like a ”conjured” environment, since we do not keep contact with actual customers but the similarity is close and has helped challenge my work principles.\\

In regards to the group work, we started out with a flexible meeting schedule, which worked most of the time but I personally prefer a more planned approach. Sometimes we did not manage to review the work of all team members, which ultimately resulted in assignments with missing or inconsistent content. Consequently, we ended up using Trello to achieve a better overview along with a time schedule that all team members agreed with. This helped improve the cooperation and coordination of the project dramatically.\\

As in the prior projects, we used version control (Git) allowing us to work simultaneously on the mandatory assignments and code. This worked very well but required a strategy on how to use Git in order to avoid merge conflicts. Thus, we thoroughly went through on how to use branches and specific naming conventions when working on separate parts of an assignment or code project. I think we should have done this from the start, instead of having many merge conflicts requiring refactoring. This along with proper coding conventions minimize the overall time spend refactoring and makes what we produce more consistent and correct from the start. Consequently, I personally wrote a document about naming conventions and branches in Git, which has been read and understood by all team members. I had a bad experience with the lack of these agreements in the First Year Project were people would have different opinions about documentation, placement of brackets and version control. This ultimately resulted in many extra hours of work close to the deadline, which could have been avoided.\\

The work load was equally distributed among team members and I did not find any personal issues working together. This is based on previous experience working with the same people. Thus, we already know our different personalities and have thoroughly established the different skills that we have to complement each other. The only new conflict we had in this specific project was based on disagreements about time and priorities. In other words, some members were fine working on the project all days of the week while some members preferred a clearer separation between work related activities and social life. We had to solve this by making an actual document showing which times of the week each member is available or does not want to be disturbed. This will probably affect my choice of group for the next project, since I think a general agreement of work hours and time spent on the project is important to establish a common goal and work ethic. 

\subsection{Individual Reflection William}
The major issues I have encountered during the project was in regards to:
\begin{enumerate}
	\item Time
	\item Understanding the application domain
	\item Working with LaTex using Version Control
	\item Partly Unstructured Work and lacking Communication
	\item Hand-In of Assignments
\end{enumerate}
In regards to working on LaTex, I found that the communication and the rules for using git was too vague. This resulted in many merge conflicts and lost data, because the guidelines were not clear enough.
The work was done without enough structured planning in the beginning, which resulted in people not being sure about who was assigned to which specific assignment. Instead people had a more general idea about which parts of the assignments they were working on. This made it hard to know exactly who was responsible for what, and so it was hard to know, who to go to when having/spotting a problem. About the communication, the lack of this (also in the beginning) resulted in missing hand ins or hand ins with missing sections or wrong/duplicated UML diagrams.
Also the documentation of the development could have been better if versioning had been done in the beginning of the project. New realeses of the program and the different design documents should have been made frequently during the work to reflect design choices and evolution of the program better.\\
Another major struggle has been understanding the application domain. In the beginning the application domain seemed very confusing, but not enough effort were made to generate questions for the domain expert to explain the domain. This was a huge mistake, which had a seriouse impact on the project during the implementation, because the development of the program went in a wrong direction. This resulted in a wrong implementation, which had to be chaged quickly 2 days before the handin deadline. Because of this no functioning implementation of the program was ready for handin.

\subsection{Challenges in the Cooperation and Coordination}
Some troubles were encountered in accordance to when and how much to work. This was because there were conflicting opinions about how or even if it should be acceptable to declare oneself for unreachable because one would like to separate the study from the social life.
\pagebreak

\subsection{Individual Reflection Dennis}
\section{Challenges while working on my part of the project}
One of the greatest challenges I encountered regarding the project was to have an understanding on the application domain but also an understanding on how the requirement analysis document and system design document was to be written. This resulted in some sections being incorrectly written and was consequently required to be rewritten.\\\\
Final edition\\
As written previously the greatest challenge that I encountered throughout the project, was indeed the understanding of the application domain. This resulted in the program not being implemented as required. However, after I had a personal meeting with Steven and Paolo I was then truly able to understand the application domain which I shared with the group. Yet, this was unfortunate too late. However, I did manage to create and implement a study configuration UI that was to reflect the actual process of defining a study. Taken the UI into consideration, being solely responsible for creating a UI while also learning a new framework was also really, though, but I did get out from the process.

\section{Impediments with regards to the team cooperation}
Some impediments I encountered was the lag of communication and team planning at the initial end of the project. We did not delegate the work probably and that resulted in some group members having a greater workload than others. The lag communication resulted in some work being incorrectly made or some required work resources not being shared correctly so others can continue their activity. By example, when some UML was made one may forget to share these to the whole group so they can proceed with their section/activity.

Another thing regarding communication is the fact that we did not agree on common terminologies. Therefore, one may use one terminology while another uses a different terminology. Consequently, the report ended up being inconsistent and was therefore required to be fixed. This could all be avoided if we had a clearly communication on what was to be done and how we achieved\\\\
Final edition \\
Something I really is missed was to meet up with my group in the weekdays. This would allow us to have a deeper discussion on issues instead of going around with them alone. I'm not implying that the others didn't work. They really did and I appreciate it, but since we only meet two times a week while also being confused on the application domain was not really optimal regarding creating a proper program.


\section{Retrospectively change something with regards to my approach to the cooperation within the team}
I would have been clearer on my communication to avoid misunderstandings. I would also change the initial approach of the project in that sense of better planning and group coordination. It may have yielded greater results if we all had researched a bit more on the application domain, but also, how the requirement analysis document and system design document was to be written, so we could have avoided aimlessly assumptions on how they were to be made.\\\\
Final edition \\
Reflecting on the written above I did improve my communication. I was able to explain the application domain for the whole group which did help to steer development 
in the right direction. Nonetheless, I would in the future utilize TA's and other experts if I sense the slightest form of misconception because going through all of this again is not worth anyone's time and energy.

\section{Cooperation improvement throughout the project}
The group did, fortunately, mature throughout the progression of the project. The group had created fixed work schedules and additionally utilized a Trello board. This has yielded great results in terms of planning and communication. The fixed work schedules enable us to plan and propose what task has to be prioritized and accomplished while a Trello board allows the group to evenly distribute the workload among the members. If the group did this from the beginning, countless work hours and resources could have been allocated on some other sections of the project. Nevertheless, the group had apprehended this issue and has in that sense improved.\\\\ 
Final edition \\
Improvements did occur as mentioned above. Also, we ended up communicating more but that was, unfortunately, near deadline.


\pagebreak
\subsection{Individual Reflection Jacob}

\section{Individual Reflection Jacob}
The biggest challenges I have faced in this project has without a doubt been the UML diagrams and understanding the the terminology used by the user. It was especially hard to understand the users needs and what was required of our solution. Not properly understanding the user domain made it especially hard to create scenarios and use cases which I struggled with. This could of course have been avoided by asking the client more questions, but regrettably. //
Furthermore. In the code implementation we experienced once again our knowledge of of the application domain was lacking which lead to last minute changes to the program, redesigned  tests and people were moved from developing on the project to patch-working, which hurt our ability to deliver a satisfying program.
\subsection{Teamwork}
I think the teamwork in this group has progressed smooth and without too many hiccups. There has been incidences in which the group had to deal with team members disappointing each other, but each and every time it has been resolved swiftly and without trouble. I think we have avoided a great deal of internal conflict by defining office hours in which team members could be contacted in. This has greatly helped group members with different sleeping patterns to separate work from spare time.
\subsection{Retrospective}
I think we could have saved ourself from a lot of hurt by implementing and experimenting with methodologies like Scrum earlier in in the process. I noticed we became much more organised after experimenting with a dedicated task board to visually represent our current progress. From my own experience, I can highly recommend this course to teach these methodologies earlier in the course.