\subsection{Organising the Project}
<<<<<<< HEAD
This is a short description of the various group structures we have used to organise our group throughout the BDSA course at ITU. To establish a common ground and similar expectations for the project, we reused a group contract from prior work. The contract provided a relaxed and open minded work environment. However, we still felt a need for a clearer distinction between group work and private life, which lead to an official definition of meeting hours. In other words, each group member specified in which time span he is available and vice versa. This allows us to avoid trespassing on group member's spare time. \\  
Initially, we kept the meetings informal without a Mediator,  but with a Note Taker. A Mediator would only get elected upon sudden conflicts that required a more planned approach. 
\\ Even though consensus was reached on the group contract, we still lacked a dedicated strategy for the weekly assignments. Tasks were randomly delegated to group members without any direct understanding of the actual required workload of the task. Consequently, an unbalanced work distribution sometimes occurred among team members, which ultimately led to otherwise avoidable group conflicts. 
\\
To solve this problem, we iteratively improved our workflow by experimenting with  different techniques, e.g. Scrum. We could not implement a pure Scrum implementation due to limitations like time constraints, but features like the task board proved especially valuable. The task board helped us visualise the current tasks and quality control already solved tasks. On the task board each task starts in the \textbf{Backlog} area. Then we move the most vital tasks to the \textbf{Current Sprint} area. From here each group member is assigned to a task. When a task is completed, it is moved to the \textbf{Review} area. Then it is evaluated by other group members. The task is either approved and moved to the textbf{Done} area or failed and moved back to the \textbf{Current Sprint} area. As a result, all tasks are reviewed and continuously kept track off. 
=======
This is a short description of the various group structures we have used to organise our group throughout the BDSA course at ITU. To establish common grounds and expectations for the project, we revisited a prior group contract we all prior agreed too and choose to reestablish the contract. Our group contract provided a relaxed and open-minded work environment, but we felt we needed a sharper separation between group work and private life, which lead to our definition of office hours. Each group member specified in which timespan he was available and when he wasn't.  This allowed us to avoid trespassing on group members spare time. and kept us more motivated\\  
Like in prior groups, we kept the meetings informal without a Mediator,  but with a Note Taker. If the debates should  get heated, a Mediator would get elected.
\\While we had a group contract we all supported, we lacked a dedicated strategy to tackle our weekly assignments. Task were randomly given to group members without any real understanding the workload of the task. This resulted in unbalanced work distribution among group members which led to minor internal frustration. \\
To solve this problem we iteratively improved our workflow by experimenting with  different techniques like Scum. We could not implement a faithful Scrum implementation due to limitations like time constraints, but features like the taskboard proved especially valuable. The Taskboard helped us visualise the current task as well as quality control of the solved tasks. On the taskboard, each task would start in the \textbf{Backlog} area. Then we would move the most vital tasks to the \textbf{Current Sprint} area. From here each group member would assign themselves to a task. When a Task is completed we moved it to the \textbf{Review} area.  Here would other group members evaluate it and either approve the solution and move it to the textbf{Done} area or fail the solution and move it back to the \textbf{Current Sprint} area. This assured all items would be reviewed and not forgotten in the heat of a deadline. 
>>>>>>> origin/master


