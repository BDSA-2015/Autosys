\subsection{Retrospect}
<<<<<<< HEAD
When thinking of the group work in retrospect, it has become clear that certain things could have been done differently, which would probably have improved the communication, cooperation and efficiency in the group.
The first thing which should be mentioned is the attempt to follow the SCRUM method, which only was accomplished partially, since one of the core features (the stand up meetings) was not followed consistently by the group. If these meetings had been upheld, it would presumably have strengthened the communication. As a result, this could potentially have lead to fewer misunderstandings and less miscommunication during the work.
\\
During the beginning of the group work, Trello was not used properly, which meant that the group work did not become as structured as it could have been. By using the Trello board probably it would also have made the first issue with SCRUM easier to handle, since it would have been possible to structure the SCRUM using this tool. Also, the use of a Trello board would have made the planning and distribution of tasks a lot easier.
\\
Better communication could also have been accomplished by the use of a Trello board combined with a more structured use of the Facebook group, e.g. by setting up some guide lines on how and what to write. By scheduling strict deadlines and communicating more about them, some of the unfortunate mistakes with missing content, which happened during hand ins could have been avoided. Further, a better set of rules for VCS when writing the documents in LaTex could have prevented some critical compile errors.
\\
In the beginning, the working hours were very flexible and mostly decided based on people's job schedule. This lead to occasionally late working sessions and meetings where only parts of the group could attend. Thus, it had an impact on the stress level in the group, which was why a decision was made to make a schema containing the office hours where people could be contacted and why a planning a fixed meeting schedule for the week was made. This should have been done a lot earlier in the work process, since this initiative created a better working environment for most of the group members.
\\
Besides the group work, it has become clear that a better use of TAs and application domain specialists throughout the course would have been rewarding, because this would have resulted in a better understanding of the application domain from the beginning, which would have meant less resubmissions and a better foundation for future work. 
Some of these challenges mentioned above might originate from a delegation of tasks happening too fast, and thus the group did not always take the time to talk about the theory and establish a solid and common knowledge before beginning the work. In this way, the approach became much more practical with a "fail faster" mentality, that also had its pros because a lot of practical experience was achieved quickly. However, some resources could have been saved by using slightly more time on the theory before trying to solve the tasks.
=======
When thinking of the group work in retrospect it has become clear that certain things could have been done differently which would probably have improved the communication, cooperation, and efficiency in the group.
The first thing which should be mentioned is the attempt to follow the SCRUM method, which only was accomplished partially since one of the core features (the stand-up meetings) was not done by the group. If these meetings had been upheld it would probably have strengthened the communication which could potentially have lead to fewer misunderstandings and less miscommunication during the work.

During the beginning of the group work Trello was not used probably, which meant that the group work did not become as structured as it could have been. By using the Trello board probably, it would also have made the first issue with SCRUM easier to handle since it would have been possible to structure the SCRUM using this tool. Additionally, the use of a Trello board would have made the planning and distribution of tasks a lot easier.

Better communication could also have been accomplished by the use of a Trello board combined with a more structured use of the project's  Facebook group e.g. by setting up some guidelines for how and what to write. By scheduling strict deadlines and communicating more about them, some of the unfortunate mistakes with missing content, which happened during hand ins could have been avoided. Furthermore, a better set of rules for VCS when writing the documents in LaTex could have prevented some critical compile errors.

In the beginning, the working hours were very flexible and mostly decided based on people's job schedule. This lead to occasionally  late working sessions and meetings where only parts of the group could attend. Moreover, it had an impact on the stress level in the group, which was why a decision was made to make a schema containing the office hours where people could be contacted and why a planning a fixed meeting schedule for the week was made. This should have been done a lot earlier in the work process since this initiative created a better working environment for most of the group members.

Besides the group work it has become clear that a better use of TAs and application domain specialists throughout the course would have been rewarding, because this would have resulted in a better understanding of the application domain from the beginning which would have meant fewer resubmissions and a better foundation for future work. 
Some of these challenges mentioned above might originate from a delegation of tasks which was too fast, and thus the group did not always take the time to talk about the theory and establish a solid and common knowledge before beginning the work. In this way, the approach became much more practical with a ''fail faster'' mentality which also had its pros because a lot of practical experience was achieved quickly, but maybe some resources could have been saved by using slightly more time on the theory before trying to solve the tasks.
>>>>>>> origin/master
