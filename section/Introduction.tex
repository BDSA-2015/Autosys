
\section{Introduction}

\subsection{Purpose of the System}
Today, the amount of data has grown to a level where it is starting to challenge researchers who need to extract the best available research on a specific question. As a result, researchers apply systematic studies on big data sets where they exploit meta data to classify subsets with useful data. This requires smart data processing tools that can use meta data to narrow down relevant research data. The purpose of such a systematic review is to sum up the best available research on a specific question. This can be achieved by combining the results of several studies. In this regard, the system in this project is comprised of two parts, client and server side. Our system scope is restricted to the server side and shall support the configuration of summarized research data relevant to a given research question. 

\subsection{Scope of the System}
The server shall provide teams with tools to conduct secondary studies (SMS or SLR). It should support activities of planning and conducting a review distributed in a research team. The server shall make sure that all data is stored for use in setting up a study configuration requested by the client. The system should be able to import information from a bibtex file to a database. The reason is we want to be able to populate our database with existing data. Security matters (e.g. user authentication) are not taken into primary account due to the scope of the project. 
In order for the system to fulfill the previously described purpose, it has to support the tasks described in the "Proposed System" section. Among these, the system shall support management of distributed research systems to work on a study. The reviewers should be able to export data sets and filter them with inclusion and exclusion criteria. Finally, the system should allow specific sets of data to be reviewed and screened by specific members of a research team. 

\subsection{Objectives and Success Criteria of the Project}
The system should be easy to deploy and install. It should include an installation and user manual used to describe how to configure and prepare research papers for screening. It should be easy and quick to distribute relevant data and the overview should outcompete the one achieved in other third-party programs such as e.g. Excel. The system should define rules for which data goes to whom to achieve a succesful screening of paper and efficient data extraction. The yellow system has to provide a user interface from which the blue team can extract data based on user roles and rules. The system has succeeded if users in the blue team can query the yellow system for relevant studies and tasks based on a given study configuration. Specifically, the yellow system should collect research and aggregate stacks of research material based on a research question. The blue system should then efficiently be able to extract a subset of primary studies provided from the screening of the search hits in the yellow system. The configured data may then be used by reviewers in the blue system (visualize, sort, export and categorize). Finally, the system should be able to replicate an existing study.

\newpage

\subsection{Definitions, Acronyms, and Abbreviations}

	\begin{itemize}
	  \item Systematic Studies: methodology used to sum up the best available research on a specific research question or topic.  
	  \item Yellow system: server side also referred as "server". The main responsibility of the yellow system are to store, send, validate and manipulate data from the database and also configurate a study.
	  \item Blue system: client side also referred as "client" The main responsibility of the blue system are to visualize data from the database according to the user's demands and manage teams.
	  \item Study: Is a specific structured research area with insight of existing contributions.
	  \item Stage: is a given set of review tasks. Each stage is dependent on each order and is completed in a fixed order.
	  \item Review task: Is a task in a given study process. Such as defining research questions
	  
	\end{itemize}
	
\subsection{References}
Tell, Paolo, and Steven Jeuris. Autosys: Supporting Distributed Teams Performing Systematic Studies. 1st ed. Copenhagen: ITU, 2015. Print.

\subsection{Overview}
The rest of the document will describe the current systems available and the proposed system in our project along with the requirements collected from users, customers and stakeholders. The system requirements are used to generate system models such as potential scenarios and use cases. 
