\subsection{Designing the Global Control Flow}
When selecting components for the interface and storage subsystems of Autosys, we effectively narrowed down the alternatives for control flow mechanisms for the yellow part of Autosys. The AutosysServer is located on a web server. The AutosysServer awaits requests from the blue part of autosys or the studyConfigurationUI. For each request the AutosysServer receives, a new thread is made, which enable it to parallel handle the requests. This results in a more responsive system. By example, the AutosysServer can process and handle a given process x while another process awaits a respond from the database. However, the stumbling block of threads is the increased complexity of the system resulting from the usage of threads. To establish a sturdy design with concurrency taken into consideration, one will define the following strategy for dealing with concurrent accesses to the shared storage:
\begin{itemize}
	\item \textit{Boundary objects should not define any fields.} Boundary objects should only hold temporary data correlated with the current request in a local variable.
	\item \textit{Entity objects should not provide direct access to their fields.} All changes and accesses to a given object state should be done through dedicated methods. 
	\item \textit{Methods for accessing state in entity objects should be syncronized}. By using thread synchronization mechanism provided by C\#, only one thread can be active at a time in an access method.
	\item \textit{Nested calls to synchronized methods should be avoided.} When creating synchronized methods, one must make sure if a nested method call can lead into calling another synchronized method. The reason for this is to avoid deadlocks and must be avoided. If nested calls are unavoidable, one should either relocate the logic among methods to or impose a strict ordering of synchronized method calls.\footnote{This is deprecated since an Entity-framework database is being utilized. Therefore it is assumed that this requirement is automatically handled by the framework.}\label{nestedcallsglobalflow}
	\item \textit{Redundant state should be time-stamped} The state of an object can periodically, be duplicated. By example, two researchers may create objects with the same state and can lead to conflicts. To avoid this, objects should be time-stamped or have another unique identifier.\footnote{This is also deprecated. The reason is the same as the footnote above this one }

\end{itemize}