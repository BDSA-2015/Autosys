\subsection{Identifying and Storing Persistent Data}
\paragraph{Identifying persistent objects}\mbox{}\\

Autosys deals with three sets of objects that must be stored.  The first set (referred as metadata storage) consists of research data such as metadata on articles/papers that are uploaded to the server. The second set (referred as blue storage) consists of objects that are created and accessed by the blue part of the system(eg. Users, tasks, etc). It needs to be persistent to track the progress of the study and who is involved. The third set (referred as configuration storage) consists of data for a study configuration that are created by the study configuration UI. The data defines the study configuration such as data fields, tasks, inclusion and exclusion criteria etc.\\\\ 

Metadata storage is well defined and will rarely change during the lifetime of Autosys. These changes may occur whenever new research data has been created or removed and thus create a need for an update of the set.\footnote{This definition as been changed after the discussion with the client. It was previously assumed that the resources were to be stored as a shared resource pool were other studies were able to access. However, this assumption has been redefined so sets of resources are bound to their respected studies.  Therefore, the lifetime of these objects are not bound to the program but are rather bound to the lifetime of a study.  However, as mentioned in this section this part of the storage is well defined and will rarely change since one can only define a given set of paper to a study when it is being configured. } Objects within blue storage are managed and defined by the blue part of the system. Hence, we decide to let the blue part decide how to manage and access these persistent object through a generic interface.\footnote{After descussion with client. It should be said that tasks are created by the server and are therefore managed regarding their storage.} Configuration storage is also well defined and will not change once the study configuration has been made.

In this scope of the Autosys system, the main focus of persistent objects is set on metadata storage and study configuration storage.\footnote{In addition, taskstorage is also a primary focus.}

\paragraph{Selecting a storage strategy}\mbox{}\\
By selecting and defining a persistent storage, strategy enables us to deal with issues related to storage management. The main design goals of the yellow part of Autosys is to be reliable, scalable while also having high performance. It is therefore decided to implement a database management system since it allows concurrent queries and provide transaction mechanisms to ensure data integrity. \\\\Compared to a flat file storage method, a database management system will also scale to large installations with many researchers that are to conduct studies simultaneously. However, to allow future upgrades or changes it has been decided that the storage strategy is not solely dependent on a database management system. Thus, the storage subsystem will provide an abstract interface that enables other kinds of storage to be coupled. The users of Autosys are not able to change the storage, however if an update is ever needed for the storage strategy, a system administrator are able to do so. Since the system is being developed in a Microsoft environment, the group has decided that Entity Framework, which is provided by Microsoft, will be utilized. The reason for this is it enables the group to easily define and create a database that can be used with their system.