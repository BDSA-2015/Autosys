\subsection{Purpose of the System}
In Autosys, the main client is the SystemReviewClient who needs to set up a study. A study consists of a set of resources on papers,  one or several phases, a set of data fields, set criteria and a set of study participants. The papers that are bound to a study are being filtred with a set of inclusion- and exclusion criteria. These criteria are based on data fields such as year, title and so forth. A \textbf{PHASE} will consist of researchers that have various roles. One can either be a reviewer or validator. The main responsibility for a reviewer is to review a given number of papers from a set of visible data fields and requested data fields that are defined in each phase. Visible data fields determines what the reviewer will see from a given paper. By way of example, if one had "year" as visible data field one will only see a given year when the given paper is to be reviewed. Requested data field are datafields that are defined but has not been answered. By way of example, a requested data field could be defined as a data field about whether a paper is about soft engineering or not. The reviewer will answer this based on the given set visible data fields. This process is regarded as a review task. Each review task is given to reviewers based on their defined workload. A workload defines on how many papers they are to review. When some task are done and they have conflicting data. By way of example, Reviewer A has returned a task request that differs from Reviewer B's task request. The validator will then be given a task where he must validate which of the task request are accepted. The \textbf{StudyParticipants} will be working together on a \textbf{STUDY} in \textbf{TEAMS}, but multiple teams will not be working on the same \textbf{STUDY}. Furthermore \textbf{TEAMS} working on the same \textbf{STUDY} will be considered as one big \textbf{TEAM}.\\\\ The purpose.\footnote{The previous assumption on the solution domain was to let the user create a study  where he, in addition, was to establish the tasks and distribute them manually in the Study Configuration UI and then submit it to the study configuration server. This assumption was brought up to discussion with the client and was redefined so tasks were to be generated automatically by the study configuration server when he had uploaded a configured study.} of the system that yellow are to create, is to create a study-configuration UI that enables a study manager to define a study and their members, resources, phases roles and data field. The Study Configuration UI should also let the study manager manage existing studies by editing or deleting them. Another part of the system that the yellow team is responsible for is to create a Study Configuration Server that can create tasks defined in a study and distribute them to each member automatically. The task destribution workload percentage can either be defined as a fixed number or a user input. The server must also check for conflicting datafields when a given task is complete and then create conflict handling tasks for validators. The server must also be able to store studies, tasks, users and teams within database so they can be retrieved or modified at any given point of time.
