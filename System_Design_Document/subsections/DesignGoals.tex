\subsection{Design Goals}

\begin{itemize}
\item \textit{\bf{Ease of use}}: 
\\
Goal: it should be relatively easy to setup a study configuration because it makes up the foundation that all study work processes rely on. 
\\\\
Trade-off: the end user may have a low level of computer expertise potentially resulting in the wrong setup of a study.  This can happen because the user cannot find or access the ressources required for setting up a study or they become frustrated if they have to go through many windows. However, these usability traits should not compromise with the system functionalities. The system must still be sufficiently complex in order to provide a variety of ways to setup the configuration.  A primary focus on usability traits has been chosen due to the scope and time span of the project. This design goal is a refinement of the non-functional requirement  "usability" in Requirement Analysis Document  (section 3.3.1). 

\item \textit{High Reliability}: To make it possible to work with data even if you have no internet connection. Auto save of configuration data.

\item \textit{Scalability in terms of the amount of users and concurrent studies}:\\
Goal: The response time of the system must not degrade dramatically with the number of these users.\\
Trade-Off: The system is used by multiple teams with several users working on different studies. The work carried out by these users could be done concurrently and so the system will have a big workload when multiple requests are send to the server. Since the users need the study data quickly to conduct their research, it will be cumbersome if the users have to wait for a long time to get the data. Because of this the system will have to make the data quickly accessible to the users. But still the system must do this in a way which takes memory into account, so it does not require to many database resources.

\item \textit{High performance for data request processing time}: indexing in the database, 
\end{itemize}