\section{Integration testing}

Unit Integration testing is a as the name apply a tests which components are working correctly together. It assume the the individual components are working correctly and is strictly testing for compatibility errors. This test approach allows us to test interconnectivity of classes instead of viewing them as isolated systems, and as a result gives us a broader image of which systems are effected of defect sub components. 

\subsection{Different types of integration testing}
How to use integration test depends on your test methodology  (Top down or  Bottoms test strategy). 
\begin{itemize}
	\item \textbf{Top down test strategy} : implement integration testing by first testing the high level components integration and from there working downwards to the lower level components. This allows high level component and subsystems to be tested early in the development but comes with cost of delaying some test of the more complicated integrations test due to the lower level components have not been implemented yet.
	\item\textbf{Bottoms up test strategy} : implement integration testing by first integration test low level components and from there iterative integration test each ascending  components and modules integration in the system. This ensures the programs module are integration tested early in the test phase, but post pone all high level components integration tests to late in the project
\end{itemize}

\subsection{Design choice}
Given the two different test options we went with a bottoms up test approach. This is due to how we have choose to implement the system which is a iterative bottoms up approach. Since we are striving for a Test Driven Development then the lower level components are always tested first. While we are black box testing the subsystem it becomes convenient to create a integration test while we are testing the system. This will lead to the more complicated components will be tested last which will be a challenge, but hopefully our extensive work with the lower level components will ease the challenge of testing the most complicated subsystems