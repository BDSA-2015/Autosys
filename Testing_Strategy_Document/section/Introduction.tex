\section{Introduction}
We have chosen to do test driven development, because it helps to spot clunky and faulty code, which need to be refactored/improved early during implementation.
This means that the code has to be modular since it will otherwise be hard to test against.
The tests also forces you to get a clear idea of the constraints of the implementation that you are building.
Since you are often running your tests, you also strengthen your belief in the system.

The TDD has been chosen even though some cons also exist:
The maintenance of the test suite adds additional work to the implementation, but then again gives a higher belief in the functionality of the system and helps locating bugs early on, which seemed to be worth the extra work.