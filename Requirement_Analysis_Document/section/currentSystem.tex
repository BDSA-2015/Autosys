\section{Current System}
The current system is an Excel based application that contains data about contents of articles and facilitates searches on academic topics. It is possible for multiple users to contribute data and one can identify if another user is contributing data on the same article. The contributed data is mostly keywords, which makes it easier to search and compare articles. To search for articles, a user must submit keywords, which the articles will be ranked after. Articles are also ranked after the amount of keywords they have in common with other articles. The current system works but is increasingly difficult to manage. Thus, a new system which solves the negatives while keeping all of the functionality that Excel already supports is recommended. This can be done with a separation of the data and the user interface, by creating a dedicated database and a dedicated user interface. This will ensure that new functionality will not be limited by the database structure. The negatives and pros of the current system are described below:

\subsection{Pros of current system}
The current system pros are as follows:

\begin{enumerate}
	\item Excel licenses are cheap and widely used by customer base
	\item Simple and easy to augment and implement changes.
	\item Data is safe from hostile users if excel files are read only. 
	\item Easy to implement version control on .csv files. 
\end{enumerate}

\subsection{Negatives of current system} 
And the systems negatives are as follows: 

\begin{enumerate}
	\item The data structures can be hard to manage, especially as the system grows in scope and content.
	\item The system does not support multiple users editing the same files at the same time.
	\item The system interface leaves a lot to be desired.
	\item New features becomes hard to implement as the system grows.
\end{enumerate}
