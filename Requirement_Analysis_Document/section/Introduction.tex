
\section{Introduction}

\subsection{Purpose of the System}
Today, the amount of data has grown to a level where it is starting to challenge researchers who need to extract the best available research on a specific question. As a result, researchers apply systematic studies on big data sets where they exploit meta data to classify subsets with useful data. This requires smart data processing tools that can use metadata to narrow down relevant research data. The purpose of such a systematic review is to sum up the best available research on a specific question. This can be achieved by combining the results of several studies. In this regard, the system in this project is comprised of two parts, client and server side. Our system scope is restricted to the server side and shall support the configuration of summarized research data relevant to a given research question. 

\subsection{Scope of the System}
The server shall provide teams with tools to conduct secondary studies (SMS or SLR). It should support activities of planning and conducting a review distributed in a research team. The server shall make sure that all data is stored for use in setting up a study configuration requested by the client. The system should be able to import information from a BibTeX file to a database. The allows to populate the database with existing data. Security matters (e.g. user authentication) are not taken into primary account due to the scope of the project. 
In order for the system to fulfill the previously described purpose, it has to support the tasks described in the "Proposed System" section. Among these, the system shall support management of distributed research systems to work on a study. The reviewers should be able to export data sets and filter them using inclusion and exclusion criteria. Finally, the system should allow specific sets of data to be reviewed and screened by specific members of a research team. 

\subsection{Objectives and Success Criteria of the Project}
The system should be easy to deploy and install. It should include an installation and user manual used to describe how to configure and prepare research papers for screening. It should be easy and quick to distribute relevant data and the overview should outcompete the ones achieved in other third-party programs such as e.g. Excel. The system should define rules for which data goes to whom to achieve a successful screening of paper and efficient data extraction. The yellow system has to provide a user interface from which the blue team can extract data based on user roles and rules. The system has succeeded if users in the blue team can query the yellow system for relevant studies and tasks based on a given study configuration. Specifically, the yellow system should collect research and aggregate stacks of research material based on a research question. The blue system should then efficiently be able to extract a subset of primary studies provided from the screening of the search hits in the yellow system. The configured data may then be used by reviewers in the blue system (visualize, sort, export and categorize). Finally, the system should be able to replicate an existing study.

\newpage

\subsection{Definitions, Acronyms, and Abbreviations}

\begin{itemize}
	\item Systematic Studies: methodology used to sum up the best available research on a specific research question or topic.  
	\item Yellow system: server side also referred as "server". The main responsibility of the yellow system is to store, send, validate and manipulate data from the database and also configure a study.
	\item Blue system: client side also referred as "client" The main responsibility of the blue system is to visualize data from the database according to user demands and to manage teams.
	\item Study: the whole work process from initiating a research to narrowing down relevant research evidence. A study consists of different phases where data is continuously synthesized and approved by users with different roles. The end result is a final set of primary studies used to clarify the research question. 
	\item Primary and Secondary study: secondary research is defined as an analysis and interpretation of primary research. The method of writing secondary research is to collect primary research that is relevant to a given topic and interpret what the primary research found.
	\item Phase (stage): is a given set of review tasks. Each phase is dependent on each other sequentially and is completed in a fixed order. Each phase details how task requests are handled and handed out. Optionally, a user role can proceed to the next phase even though other user roles have not finished the same phase yet (e.g. two concurrent review phases). Each phase also has a defined set of visible data fields and requested data fields. These data fields  will be the template for the tasks in the phase.
	\item Workload\footnote{\footnote{This was added after a discussion with the client. This was added because tasks are now being handled by the system and not defined by users}}: Workload is a defined percentage measurement on how to populate task to the study members. By way of example, if a workload is one hundred percent for every reviewer then every reviewer will get a hundred tasks out of a hundred tasks. Another example will be that a workload has been set to fifty-fifty. This means that the workload is split. The workload is all about a balance between how many conflicts that to be produced and how much workload should be put on the members. a hundred percent workload on every member would yield greater numbers of conflicts between the tasks, but this would require every member to go through all of the papers.
	\item Task: an assignment in a given phase in a study. A task is defined by a set of visible data fields, a set of requested data fields and a type. A task type can either be a request to review and fill out data field(s) by a reviewer or a request to handle conflicting data field(s) by a validator. By way of example, a phase could involve review tasks assigned for two reviewers. A validator could then analyze any inconsistencies between the task request of both reviewers and select which of the requested data fields is appropriate for the given paper.
	\item Data Field: a data type that defines a given information on a paper. Data field can for an example be a given Author for a paper.
	\item Visible data field\footnote{This was added after a discussion with the client.}: Visible data fields are a given data field used within a task. Instead of showing every date fields for a given paper, only the visible data fields will be shown. By way of example, if 'Author' was a visible data field, this data field will only be shown when papers are being retrieved
	\item Request data field\footnote{\footnote{This was added after a discussion with the client.}}: Request data fields area data fields that require to be confirmed/reviewed. They are usually confirmed by comparing their descriptions and criteria with some visible data fields for a paper. By way of example. If a requested data field would be something about software engineering. One should check if the associated paper correspondence to this data field
	\item Research Protocol: the rules on how a given study is configured (workflow).
	\item Study Configuration: the configuration of a given study. A study configuration involves defining the research question, phases, assigning members for the study, choosing data fields and assigning roles for all members. A study configuration also contains a set of papers that are to be used. 
	\item Inclusion/-Exclusion criteria: criteria is defined in the initial study configuration and is evaluated throughout the whole lifetime of a given study. By way of example, criteria could be whether the data is from later than 2005. The criteria are used along the way to synthesize the data. As opposed to the classification criteria that is only used in the end of the study. 
	\item Classification Criteria: classification criteria is used to group data fields in a given study. By way of example, one could classify two groups for the same set of primary studies. The classification criteria determine what one is looking for in the data. It is defined in the study configuration and is used in the end of a study upon data extraction. This is used to approach or investigate the same data in different ways.
	\item User: a person who is either a manager or a researcher. A manager can create a team while also being a researcher. A researcher can only have one role in a specific phase of a study. 
\end{itemize}

\subsection{References}
Tell, Paolo, and Steven Jeuris. Autosys: Supporting Distributed Teams Performing Systematic Studies. 2st ed. Copenhagen: ITU, 2015. Print.

\subsection{Overview}
The rest of the document will describe the current systems available and the proposed system in our project along with the requirements collected from users, customers, and stakeholders. The system requirements are used to generate system models such as potential scenarios and use cases. 
