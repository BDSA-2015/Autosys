\section{\logtitle 12/03-2015} % Remember to set
\duration{9:00}{16:00}
\attend{\at}{\at}{\at}{\at}


\begin{itemize}
	\item [\textbf{Meeting pins:}]
	\item We have agreed to make shared decisions and planning in the first hour both Tuesday and Thursday. 
	\item Thor will look at Jacobs pull request and check Phase and Criteria. Thor will check Phase and Criteria, implement Tag and Entry entities and fix Foreign Key functionality in model entities using ICollections for lists and dictionaries with objects that cannot be saved directly in the database. Remember to write System Design strategy before repositories. Finally, he will make tests for storage repositories and implement repositories based on these. 
	\item Talk about branching methodology. The issue can be seen in the branch "EntityRefraction" holding too big merge conflicts.
	\item Dennis will write on usability testing. Dennis is wrapping up on UserManagement. Need to refine documentation and tests. Dennis will present AutoMapper to the rest of the group. 
	\item Merge AutoMapper BaseClass changes from Dennis and Jacob's pull request. 
	\item Talk about Object Design model. 
	\item William will first write on unit testing in testing strategy. He will finish PaperManagement. Then work on ExportManagement implementation and testing. Will begin mockups for GUI. 
	\item Jacob will write his integration testing strategy. Jacob would like to work on functionality used to search through criteria. Jacob wants to work on search functionalities using criteria. 
\end{itemize}

\begin{itemize}
	\item [\textbf{Done:}]
	\item Implemented AutoMapper with profiles for each subsystem.
	\item Wrote testing strategies. 
	\item Fixed stored entities and foreign key functionality.
	\item Added Tag and Entry entity used for BibTex parsing
	\item Refactored and cleaned up code.
	\item Bug fixing. 
	\item UserManagement tested and documented. 
\end{itemize}

\begin{itemize}
	\item [\textbf{Sprint Planning:}]
	\item UserManagement to be done. 
	\item PaperManagement to be done (not ExportManagement). 
	\item Storage repositories to be done. 
	\item Adapter between WebApi and Main Handler to be done. 
	\item MainHandler and AutoSys Server need to be finished next week. 
	\item Start looking at UI. 
\end{itemize}

\begin{itemize}
	\item [\textbf{Design choices:}]
	\item \textbf{Export}: William would like to know how we import and export BibTeX files and protocols. William suggests that we store BibTeX files on the computer as a text file, we parse them in a program as Paper objects and Paper will hold references to resources on the computer. This allows to find a resource on a computer with a key from Paper. The pdf file is sent as HttpRequest as DataStream. The tricky part is to update the pdf in the directory when the BibTeX is updated. However, this is assumed nonvital.
	\item \textbf{Import}: Should BibTex parser be customized when parsing or should it have a default setup? Do we assume that the BibTeX file is already imported from the start? Blue Client sends raw BibTeX file to Server, this should be parsed into the database. We need to take raw BibTeX file and save as Papers. Blue team will support UI to import BibTeX file to database. They need to do this with user defined tags outside the server. We accept BibTeX syntax but could allow input like "Retard" instead of "Article". The suggestion is to compare BibTeX file to default tags in the system. If no match it will not become invalid but just save the BibTeX file with this info. Undefined tags will be defined as new BibTex tags in the BibTeX parser. All well defined is saved as papers in the database. All undefined tags will be saved as new BibTeX entries in the database. Need to save new Tags as strings in the database. This allows to search for things in the database with BibTeX tags that only exist. Field entry type (Article) can just be chosen, true false. Fields that do not exist on all papers, the system needs to know if field exist. It can either compare with an existing list of fields in the database. Or it can look up in the database with Papers (could potentially be a big list). The field type could also have a reference to papers with this type. The solution is to have a Tag entity holding all fields as strings retrieved from id references in Paper. 
	\item Decision: entries and fields are saved in the database and mapped with the Paper. The paper will have an entry type (Article), a list of fields (enums restrict new fields) and resource reference int (used to find pdf). We will no longer make checks in the BibTeX file but only import the whole file. 
	\item Our system becomes more flexible by allowing users to upload BibTeX files with custom tags that fit their Research but is not generally found in BibTeX format. Tags should be stored in the database. 
	\item New entities: Entry with string EntryType (required) e.g. History article, List of Paper ids and Tag with string TagType, list of Paper ids 
	\item 
\end{itemize}