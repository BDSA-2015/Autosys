In general, we have strived to use the design principles of the Repository pattern. It is used as a layer in the code to separate the business code in our Application Logic and the database. Consequently, the business logic only accesses data stored in the database through the Storage Adapters described previously. These Adapters use the repositories on mapped objects that are ultimately passed to our version of the DbContext class, AutoSysDbModel. The AutoSysDbModel is responsible for interacting with the data objects and writing them to the database (MSSQL 2014) using concrete repositories. This has been done to isolate the data layer, support easier unit testing and allow replacing of the database technology. Also, code maintainability and readability is increased by
separating business logic from data or service access logic. \\

\noindent
In practice, the Adapters separate the logic and maps data objects to the database by using the the AutoMapper framework. Repositories are used to perform basic CRUD operations on these data objects for permanent storage in the database. The repositories then query the data source layer (database, either local production database in SQL Server 2014 or in-memory database using fake connection in the Effort Framework for integration tests). The repositories do not map the
data from the stored model entities in the Storage package but AutoMapper is used to map the data source to a business logic entity in the StorageAdapter classes in the ApplicationLogic. These adapter classes are likewise used to store persist changes in the business entity to an Entity Framework database. Thus, the repository separates our business logic from the interactions with the
underlying database.\\

\textbf{Pros}\\
Using the Repository design pattern has centralised our data logic and helped
create a flexible architecture that can be adapted as the overall design of
the application evolves. Without the repositories, we would have to access data
sprinkled throughout our codebase. By way of example, updating a table would
require to find every spot in our code that relies on it and update all of them.
With the repository pattern we only need to update a single class and the rest
of the code is unaffected. This corresponds well to the Single Responsibility
Principle stating that each thing should have a single reason for change. The
repository helps us isolate changes and limit the scope of an entity model to a
single level of abstraction. Ultimately, increasing readability and maintainability.\\

\textbf{Cons}\\
The only downside is the fact that our lacking knowledge on the domain objects
has forced us to refactor the data representation of all stored model entities numerous
times. As opposed to having a clear understanding of the business logic
and mapping these objects correctly to the permanent storage from the start.
Also, we struggled choosing whether the repositories should be async or not.
Consequently, we chose async assuming the program could possibly run with a
hosted database. In that case, async maks it possible to use CPU resources that
are otherwise not usd while threads are waiting for an answer from database
queries. Finally, we did not have a clear opinion on when to dispose the DbContext
used to interact with our database. Initially, we disposed the context for
each database method call in the repositories. Instead, we now implement IDisposable
throughout all interfaces between the WebApi and the Storage layer to
dispose after each client request is invoked and completed instead.\\

\textbf{Conclusion of the Repository Design}\\
We have performed numerous design choices in the actual implementation of
the repository pattern. Mostly, the challenges considered have been whether we
should use a generic repository or several concrete entity-related repositories
and mapping the entities. We have chosen latter to achieve a better separation of concerns
and reduce unnecessary unit testing complexity. As a result, we end up
with more repository classes but also a significant readability and maintainability
improvement and easier unit testing. The repository methods only need to
change if the flow of data access changes and it is not impacted by the code in
the business logic. Thus, working with a repository built on SOLID adds a lot
more speed to future additions. 