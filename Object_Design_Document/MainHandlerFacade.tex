Multiple handler classes were implemented within their own well defined area of responsibility in an attempt to uphold the Single Responsibility principle. However, having these in different packages also complicated the understanding of the final system as opposed to having all Handlers gathered in one place. 
In this case, a Facade Pattern might have been useful, since it would make it possible to wrap the complex set of handlers and offer a simple way to access the required handler functionalities using a MainHandler Facade as a facilitator.
The use of the Facade pattern has not been entirely achieved in our code structure. We do not offer an overall interface to abstract away the concrete subsystem package Handlers. However, the principle of having one facilitating class providing the functionality of the subsystem classes is respected.
Also, the MainHandler facade could decouple the subsystem of handlers from their referenced classes. 
The limited functionality provided through the facade pattern would reduce the program flexibility as opposed to using all Handlers independently. However, we have determined that the notion of a facade using different underlying functionalities from within a specific working area created a modular and easy-to-maintain code structure. 

