\subsection{Interface documentation guidelines}
We have strived to follow a defined set of documentation guidelines in order to have a consistent codebase structure and design. Also, this is used to improve communication and understanding between us as a team of developers. \\

\textbf{Base Guidelines}

\begin{itemize}
	\item All classes are named with singular nouns.
	\item All classes and their respective methods are to be documented before they are implemented. The documentation must be precise in regarding to what the given class or method is responsible for. This will enable developers to work more efficiently without the need to spend time on analyzing poorly written documentation or the implementation itself.
	\item All methods must be defined in a way so that they only have one responsibility. Long methods should be refactored into helping methods or moved to other appropriate sections of the system.
	\item All non-trivial parameters and returns are to be documented.
	\item All method signatures must be written with camel-case and they must follow C\# formation standards. This improves readability for other C\# developers that are to read the code.
	\item Errors must be returned as exceptions and propagated to their appropriate layers. One should avoid catching general exceptions since this may mask problems that have nothing to do with the system. Instead, one must catch specific exceptions that may be relevant to a given process as opposed to just catching a general 'Exception'. By way of example, a specific exception like 'NonExistantException' would be more descriptive than a general 'Exception' if a user tries to delete a non-existent study. 
	\item One should use generic classes where seemed fit. By way of example, one should utilize an interface such as IList<object> as replacement of a concrete implementation such as 'List<object>'. As a result, it becomes easier to refactor the code since it is not tightly coupled to some concrete class.
	\item Dependency injection must be considered when creating new classes and methods. System modularity will make it easier to extend or refactor the system.
	
\end{itemize}

It should be noted that the group has decided to follow these base guidelines as much as possible. However, it should be considered that parts of the project do not live up to these expectations. Seemingly, we have strived to bring these up in this document where certain design patterns have been brought up to discussion. Areas within the codebase structure that do not apply these guidelines correctly should be taken into account and discussed in the rest of this document.