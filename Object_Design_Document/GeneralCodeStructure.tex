In general, we strived to implement the system with a 3-tier structure in our code skeleton. Each tier layer represents the Web Api, Application Logic and Storage as described in the System Design Document. This hierarchical structure has been used to ensure a modular and flexible system that meet the requirements of the SOLID principles. The Web Api layer should then communicate with the Application Logic layer and the Application Logic layer with the Storage layer. Finally, the Study Configuration UI should communicate with the AutoSys server in the Application Logic Layer. However, the final system code skeleton did not live up to these requirements. Firstly, our lack of knowledge on the application domain affected the Application Logic layer that has not been implemented properly. By way of example, we did not have a clear definition of different aspects in the study configuration work flow such as; task generation, visible and requested fields and workload distribution strategies. Consequently, our application logic is lacking core functionalities that should otherwise have been implemented. Achieving this knowledge earlier along with a correct mapping of the expected data objects in the Web Api would have ensured a correct communication between the different layers. The following chapters have been written to describe our design intentions throughout the project. Overall, we wished to offer the Web Api a "facade" in the Application Logic that would correctly assign different tasks to their respective subsystems and their field of operation. Data objects would then be mapped and written to the database using different Storage adapters corresponding to their corresponding repositories and entities.